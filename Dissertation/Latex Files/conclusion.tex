\chapter{Conclusion}
\section{Project Outcomes}
When I began this project I had a reasonably clear idea of what I wanted the project to be;

\begin{itemize}
    \item A job site aimed specifically at software developers which hosted software development jobs.
    \item The user should be able to log in securely using a username and password.
    \item The user should be able to search for jobs by a number of different methods.
    \item The user should be able to post their own jobs and delete or update them.
    \item The front end website should be visually appealing with a handful of neat features.
    \item The API should be RESTful.
    \item The application should be hosted online and accessible at all times.
    \item The job data should be deployed on a cloud database.
    \item There should be tests verifying the integrity and solidity of my front-end application.
\end{itemize}

I feel I achieved most of the goals I had when I set out with this project. Of course, there is plenty of limitations and room for improvements, however I do feel that I have gained a lot from the project and if I was to go back and start it again, would still choose the technologies that I did.

I created what I believe to be, a decently sized project for a one person group. I believe in covering the different parts of the stack I was exposed to a number of different technologies and thus gained valuable experience in full stack development. It was very important for me from the beginning that it was a full stack application and thankfully I achieved that.

\subsection{My Findings}
I learned a number of things in undertaking this project and will review some below:

\begin{itemize}
    \item React is a very enjoyable technology to use and I will certainly look to improve upon it further in the future.
    \item Undertaking a full stack application solidified my desire to work with web applications in the future.
    \item It is very important to adequately plan out a project of this scope and hold yourself to deadlines and targets.
    \item Testing should be done alongside development to ensure bugs are caught early and code is clean and reusable.
    \item Project management is a very important skill to have when working with multiple different projects. It is essential that you dedicate time every week to a project, even if other ones are more pressing. Even spending an hour or two a week on a project that isn't as pressing as another at the time, you ensure you retain key information and continue learning about the technology.
\end{itemize}
\section{Summary}
% Mention how it went, what you learned, what is it rewarding, challenges faced etc, scope big enough? 
In conclusion I must say that overall I have genuinely found the project to be very rewarding. It helped me to gain a decent grasp on a number of new technologies as well as forced me to learn how to manage a larger project than I'm used to doing over a large time-frame. I was forced to comprehensively research multiple technologies and find pros and cons to using each one. This project also demanded I set out a clear plan of action for what I wanted to achieve and within what time-frame.

I think the main goals of this module were to learn how to manage a demanding project over a large period of time and to showcase the ability to set out and learn new technologies without being taught them by a lecturer in class. I do believe I have achieved both of these and had a lot of fun along the way. This project has confirmed for me what I want to do after college and I'm very glad I made the decisions I did. Naturally, not everything went perfectly as I would have liked but the shortcomings I faced have taught me a lot going forward.